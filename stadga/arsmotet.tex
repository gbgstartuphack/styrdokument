\section{Årsmötet}
\subsection{Befogenheter} 
Årsmötet är föreningens högsta beslutande organ. 

\subsection{Sammanträden} 
Det skall hållas ett ordinarie årsmöte varje år innan den 31 mars. Utöver detta kan extrainsatta årsmöten hållas. 

\subsection{Utlysande}
\subsubsection{Kallelse} 
Årsmötet sammanträder på kallelse av styrelsen. 

\subsubsection{Utlysningsrätt}
Rätt att hos styrelsen begära utlysande av årsmöte tillkommer:
\begin{itemize}
	\item Medlem i styrelsen  
	\item Medlem, förutsatt att minst 10\% av föreningens medlemmar stödjer förslaget .
	\item Efter utlysning skall årsmöte hållas inom 15 arbetsdagar
\end{itemize}

\subsubsection{Utlysning}
Föreningens styrelse beslutar om tid och plats för föreningens årsmöte. Årsmötet
skall utlysas minst 15 arbetsdagar i förväg och anslås enligt \ref{sec:protokoll:anslagning}.

\subsection{Åligganden}
Följande punkter måste behandlas på det ordinarie mötet:
\begin{itemize}
	\item Mötets öppnande
	\item Mötets behörighet
	\item Val av mötets ordförande
	\item Verksamhetsberättelse
	\item Ekonomisk berättelse
	\item Revisorernas berättelse
	\item Verksamhetsplan
	\item Budget
	\item Ansvarsfrihet för förra årets styrelse
	\item Val av styrelse
	\item Val av revisorer
	\item Fastställande av medlemsavgift
	\item Övriga frågor
	\item Mötets avslutande
\end{itemize}

\subsection{Omröstning}
\subsubsection{Omröstningsförfarande}
Omröstning skall ske öppet, om ej sluten votering begärs. Vid lika röstetal har mötesordföranden utslagsröst utom vid personval då lotten avgör.

\subsubsection{Beslut}
Om inget annat anges antas samtliga beslut med enkel majoritet. 

\subsection{Närvaro-, yttrande-, förslags- och rösträtt}
\subsubsection{Närvaro- och yttranderätt tillkommer}
\begin{itemize} 
	\item Medlem.
	\item Av mötet adjungerade icke medlemmar.
\end{itemize}

\subsubsection{Förslags- och rösträtt tillkommer}  
\begin{itemize} 
	\item Endast medlemmar.
\end{itemize}

\subsection{Protokoll} 
Årsmötesprotokoll skall justeras av två av mötet valda justeringspersoner. Justerat protokoll skall korrekt anslås, enligt \ref{sec:protokoll:anslagning}, senast 15 arbetsdagar efter mötet.
