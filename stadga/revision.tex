\section{Revision och ansvarsfrihet}
\subsection{Revisorer}
\subsubsection{Inval}
Årsmötet utser revisor med uppgift att granska föreningens verksamhet och ekonomi under verksamhetsåret, val av revisor måste uppfylla en av följande punkter:

\begin{itemize}
	\item Två revisorer utses.
	\item En revisor samt en suppleant utses.
\end{itemize}

\subsubsection{Förutsättningar}
Föreningens revisorer kan ej inneha post i styrelsen. Ej heller vara gift, sambo, nära släkt eller på annat sätt i risk att vara i jäv med styrelsen.

\subsubsection{Åligganden}
Det åligger revisorerna att korrekt anslå, enligt \ref{sec:protokoll:anslagning}, revisionsberättelser senast tre arbetsdagar före ordinarie årsmöte.

\subsubsection{Revisionsberättelsen}
Revisionsberättelsen skall innehålla yttrande ifråga om ansvarsfrihet för berörda personer.

\subsection{Revision}
Räkenskaper och övriga handlingar skall vara revisorerna tillhanda senast 15 arbetsdagar före årsmötet.

\subsection{Ansvarsfrihet}
Ansvarsfrihet är beviljad berörda personer då årsmötet fattat beslut om detta.
